\documentclass[12pt]{article}
\usepackage{color,fancyhdr,ifthen,amssymb,amsfonts,amsmath,enumerate}
\pagestyle{fancy}
\setlength{\topmargin}{-.5in}
\setlength{\textheight}{9in}
\setlength{\oddsidemargin}{0in}
\setlength{\evensidemargin}{0in}
\setlength{\textwidth}{6.5in}
\setlength{\headwidth}{\textwidth}
\parindent=0in
\newcounter{questionNumber}
\setcounter{questionNumber}{1}
\newcommand{\headandfoot}[3]{\lhead{#1}\chead{#2}\rhead{\ifthenelse{\isodd{\thepage}}{Dr.~Casper {\hspace{.25in}}}{}}
\lfoot{}\cfoot{\thepage}\rfoot{#3}}
\linespread{1.75}
\newcommand{\question}[6]{\noindent\thequestionNumber.\quad #1\\[.1in]
(a) #2 \\
(b) #3 \\
(c) #4 \\
(d) #5 \\
(e) #6
\vspace{.25in}\addtocounter{questionNumber}{1}}
\headandfoot{Calculus II}{Quiz 2}{}
\begin{document}
Consider the region bounded by the curves $y^2+x=1$ and $y^2-x=1$.
\begin{enumerate}[\textbf{Problem} 1.]
\item Draw a sketch of the area between the curves.
\vspace{2in}
\item Set up an integral whose value is the area between the curves.  Do not evaluate.
\vspace{2in}
\item 
The above region is rotated around the line $y=2$.
Use the \textbf{shell method} to set up an integral describing the volume of the resulting solid of revolution.
Do not evaluate.
\end{enumerate}

\end{document}

