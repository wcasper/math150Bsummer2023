% Exam Template for UMTYMP and Math Department courses
%
% Using Philip Hirschhorn's exam.cls: http://www-math.mit.edu/~psh/#ExamCls
%
% run pdflatex on a finished exam at least three times to do the grading table on front page.
%
%%%%%%%%%%%%%%%%%%%%%%%%%%%%%%%%%%%%%%%%%%%%%%%%%%%%%%%%%%%%%%%%%%%%%%%%%%%%%%%%%%%%%%%%%%%%%%

% These lines can probably stay unchanged, although you can remove the last
% two packages if you're not making pictures with tikz.
\documentclass[11pt]{exam}
\RequirePackage{amssymb, amsfonts, amsmath, latexsym, verbatim, xspace, setspace}
\RequirePackage{tikz, pgflibraryplotmarks}

% By default LaTeX uses large margins.  This doesn't work well on exams; problems
% end up in the "middle" of the page, reducing the amount of space for students
% to work on them.
\usepackage[margin=1in]{geometry}
\usepackage{enumerate}
\usepackage{amsthm}

\theoremstyle{definition}
\newtheorem{soln}{Solution}

% Here's where you edit the Class, Exam, Date, etc.
\newcommand{\class}{Math 150B Section 1}
\newcommand{\term}{Summer 2023}
\newcommand{\examnum}{Final Exam}
\newcommand{\examdate}{August 4, 2023}
\newcommand{\timelimit}{2 Hour 20 Minutes}
\newcommand{\ol}[1]{\overline{#1}}

% For an exam, single spacing is most appropriate
\singlespacing
% \onehalfspacing
% \doublespacing

% For an exam, we generally want to turn off paragraph indentation
\parindent 0ex

\begin{document} 

% These commands set up the running header on the top of the exam pages
\pagestyle{head}
\firstpageheader{}{}{}
\runningheader{\class}{\examnum\ - Page \thepage\ of \numpages}{\examdate}
\runningheadrule

\begin{flushright}
\begin{tabular}{p{2.8in} r l}
\textbf{\class} & \textbf{Name (Print):} & \makebox[2in]{\hrulefill}\\
\textbf{\term} &&\\
\textbf{\examnum} & \textbf{Student ID:}&\makebox[2in]{\hrulefill}\\
\textbf{\examdate} &&\\
\textbf{Time Limit: \timelimit} % & Teaching Assistant & \makebox[2in]{\hrulefill}
\end{tabular}\\
\end{flushright}
\rule[1ex]{\textwidth}{.1pt}


This exam contains \numpages\ pages (including this cover page) and
\numquestions\ problems.  Check to see if any pages are missing.  Enter
all requested information on the top of this page, and put your initials
on the top of every page, in case the pages become separated.\\

You may \textit{not} use your books or notes on this exam.
You may use a double-sided, hand-written note sheet and a basic calculator.

You are required to show your work on each problem on this exam.  The following rules apply:\\

\begin{minipage}[t]{3.7in}
\vspace{0pt}
\begin{itemize}

%\item \textbf{If you use a ``fundamental theorem'' you must indicate this} and explain
%why the theorem may be applied.

\item \textbf{Organize your work}, in a reasonably neat and coherent way, in
the space provided. Work scattered all over the page without a clear ordering will 
receive very little credit.  

\item \textbf{Mysterious or unsupported answers will not receive full
credit}.  A correct answer, unsupported by calculations, explanation,
or algebraic work will receive no credit; an incorrect answer supported
by substantially correct calculations and explanations might still receive
partial credit.  This especially applies to limit calculations.

\item If you need more space, use the back of the pages; clearly indicate when you have done this.

\item \textbf{Box Your Answer} where appropriate, in order to clearly indicate what you consider the answer to the question to be.
\end{itemize}

Do not write in the table to the right.
\end{minipage}
\hfill
\begin{minipage}[t]{2.3in}
\vspace{0pt}
%\cellwidth{3em}
\gradetablestretch{2}
\vqword{Problem}
\addpoints % required here by exam.cls, even though questions haven't started yet.	
\gradetable[v]%[pages]  % Use [pages] to have grading table by page instead of question

\end{minipage}
\newpage % End of cover page

%%%%%%%%%%%%%%%%%%%%%%%%%%%%%%%%%%%%%%%%%%%%%%%%%%%%%%%%%%%%%%%%%%%%%%%%%%%%%%%%%%%%%
%
% See http://www-math.mit.edu/~psh/#ExamCls for full documentation, but the questions
% below give an idea of how to write questions [with parts] and have the points
% tracked automatically on the cover page.
%
%
%%%%%%%%%%%%%%%%%%%%%%%%%%%%%%%%%%%%%%%%%%%%%%%%%%%%%%%%%%%%%%%%%%%%%%%%%%%%%%%%%%%%%

\begin{questions}

\addpoints

\question[10]\mbox{}

\begin{enumerate}[(a)]
\item  
Set up an integral for the volume of the solid obtained by rotating the region bounded by the curves 

$$y=x^3,\ \ y=0, \ \ x = 1,\ \ x= 4$$

around the $y$-axis, using the shell method.

\vspace{3in}
\item

Set up an integral for the volume of the solid obtained by rotating the region bounded by the curves

$$y=8-x^2,\ \ y=4 $$

around the $x$-axis.


\end{enumerate}

\newpage
\question[10]\mbox{} 

Consider the polar function
$r=\theta^2,\quad 0\leq \theta\leq 2\pi$

\begin{enumerate}[(a)]
\item Create a plot of the polar function.
\vspace{3in}
\item Set up an integral for the length of the curve above.
\vspace{2in}
\item Evalulate the integral from (b).
\end{enumerate}

\newpage
\question[10]\mbox{} 

Calculate each of the following integrals

\begin{enumerate}[(a)]
\item $\int \frac{1}{\sqrt{4-(x-3)^2}}dx$

\vspace{4in}
\item $\int xe^{2x}dx$

\end{enumerate}

\newpage
\question[10]\mbox{} 

Calculate each of the following integrals

\begin{enumerate}[(a)]
\item $\int \frac{4}{(x-1)^2(x+1)}dx$

\vspace{4in}
\item $\int_0^{\pi/4} \sin^8(x)\cos^3(x)dx$

\end{enumerate}

\newpage
\question[10]\mbox{} 

A spherical tank with a radius of $3$ meters is completely filled with water.  How much work is required to pump the water out of a pipe which extends vertically one meter out of the top of the tank?  [Remember: the density of water is $1000$ kg/m$^3$ and gravitational acceleration is $9.8$ m/s$^2$.

\newpage
\question[10]\mbox{} 

Determine carefully the limit of each of the following sequences.

\begin{enumerate}[(a)]

\item

Find the limit of the following sequence, or show that it does not exist.

$$\lim_{n\rightarrow\infty} \frac{n^2+2}{\sqrt[3]{n^6+3n+4}}$$

\vspace{3in}
\item

Suppose that we want to evaluate the limit of the sum 

$$\sum_{n=2}^\infty \frac{1}{n\ln(n)^3}$$

with accuracy up to $4$ decimal places.  Calculate how many terms we should sum (you don't need to actually do the sum).  [Hint: use the Integral Test estimate for the tail]

\end{enumerate}


\newpage
\question[10]\mbox{} 

Determine if each of the following series are absolutely convergent, conditionally convergent, or divergent.
Carefully justify your answer.

\begin{enumerate}[(a)]
\item  $$\sum_{n=0}^\infty \frac{5^n+2}{4^{n}+3}$$
\vspace{3in}
\item  $$\sum_{n=1}^\infty (-1)^n\frac{1}{2n+3}$$
\end{enumerate}

\newpage
\question[10]\mbox{} 

Determine if each of the following series are absolutely convergent, conditionally convergent, or divergent.
Carefully justify your answer.

\begin{enumerate}[(a)]
\item  $$\sum_{n=0}^\infty \frac{2^n}{n^{3n}}$$
\vspace{3in}
\item  $$\sum_{n=1}^\infty \frac{n^3+n+1}{n^5+2n+4}$$
\end{enumerate}

\newpage
\question[10]\mbox{} 

Find the radius of convergence and the interval of convergence of the series

$$\sum_{n=1}^\infty \frac{x^{2n}}{n4^n}.$$

\newpage
\question[10]\mbox{} 


\begin{enumerate}[(a)]
\item  Find the Taylor series of $e^{-x^2}$ based at $x=0$
\vspace{4in}
\item  Find the Taylor series of $\int e^{-x^2}dx$
\end{enumerate}



\end{questions}

\end{document}

